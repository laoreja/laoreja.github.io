%%%%%%%%%%%%%%%%%%%%%%%%%%%%%%%%%%%%%%%%%
% Medium Length Professional CV
% LaTeX Template
% Version 2.0 (8/5/13)
%
% This template has been downloaded from:
% http://www.LaTeXTemplates.com
%
% Original author:
% Trey Hunner (http://www.treyhunner.com/)
%
% Important note:
% This template requires the resume.cls file to be in the same directory as the
% .tex file. The resume.cls file provides the resume style used for structuring the
% document.
%
%%%%%%%%%%%%%%%%%%%%%%%%%%%%%%%%%%%%%%%%%

%----------------------------------------------------------------------------------------
%	PACKAGES AND OTHER DOCUMENT CONFIGURATIONS
%----------------------------------------------------------------------------------------

\documentclass{resume_cmu} % Use the custom resume.cls style

%\usepackage[left=0.75in,top=1in,right=0.75in,bottom=0.6in]{geometry} % Document margins
\newcommand{\tab}[1]{\hspace{.2667\textwidth}\rlap{#1}}
\newcommand{\itab}[1]{\hspace{0em}\rlap{#1}}


%----------------------------------------------------------------------------------------
%	SEFL-DEFINED PACKAGES AND OTHER DOCUMENT CONFIGURATIONS
%----------------------------------------------------------------------------------------
\usepackage{verbatim}
\newcommand{\tabincell}[2]{\begin{tabular}{@{}#1@{}}#2\end{tabular}}
\usepackage{tabularx,booktabs} 

%----------------------------------------------------------------------------------------

\name{GU, XIUYE} % Your name
\address{Room 201, Unit 3, Building 5, Siji Jiangnan Court \\ Jiashan, Zhejiang, 314100, P. R. China} % Your address
\address{+86-15700080187 \\ https://laoreja.github.io \\ xiuyegu@163.com} % Your phone number and email


\begin{document}
% In CMU version, remove this!
% In princeton, machine learning first.
\begin{comment}
\begin{rSection}{Research Interests}
%Computer Vision, high level vision tasks, scene understanding. Machine learning, deep learning.
\vspace{0.5em}
\begin{rList}
\item {\bf Computer Vision:} scene understanding and image retrieval particularly.
\item {\bf Machine Learning:} deep learning, approximate nearest neighbor search.
\end{rList}
\end{rSection}
\end{comment}

%----------------------------------------------------------------------------------------
%	EDUCATION SECTION
%----------------------------------------------------------------------------------------
\begin{rSection}{Education Background}

\begin{rSubsection}{{\bf Zhejiang University}, Zhejiang, PRC}{}{Bachelor of Engineering in Computer Science expected in June 2017}{Sept. 2013 -- Present}
\item GPA: 93/100 (3.97/4.0), the third year GPA: 94/100 (4.0/4.0); {\bf Rank 1/189}.
\end{rSubsection}

\begin{rSubsection}{{\bf University of California, Davis}, CA, USA }{}{Global Research Experience in Advanced Technologies Program}{July 2016 -- Sept. 2016}
\item GPA: A (five letter grades).
\end{rSubsection}
\vspace{-0.5em}

\end{rSection}


%----------------------------------------------------------------------------------------
%	RUBLICATION SECTION
%----------------------------------------------------------------------------------------
\begin{rSection}{Publications}
\vspace{0.5em}
\begin{rList}
\item {\bf Xiuye Gu*}, Chaoqi Wang*, Cong Fu, Deng Cai. {\em A Revisit on Binary Code Learning for Large-scale Content Based Image Retrieval.} The $30^{th}$ IEEE Conference on Computer Vision and Pattern Recognition ({\bf CVPR}), 2017 (* indicates the co-first authorship). Under review. 
\item Maheen Rashid, {\bf Xiuye Gu}, Yong Jae Lee. {\em Interspecies Knowledge Transfer for Facial Keypoint Detection.} The $30^{th}$ IEEE Conference on Computer Vision and Pattern Recognition ({\bf CVPR}), 2017. Under review.
\end{rList}
\end{rSection}

%----------------------------------------------------------------------------------------
%	RESEARCH EXPERIENCE SECTION
%----------------------------------------------------------------------------------------

\begin{rSection}{Research Experience}

{\bf Research Intern -- University of California, Davis} \\
\vspace{0.5em}
\textsl{Advisor: Prof. Yong Jae Lee} \\
{\bf Interspecies Knowledge Transfer for Facial Keypoint Detection} \hfill \textsl{July 2016 -- Nov. 2016}\smallskip
\begin{rList}
\item Proposed a novel deep learning method for localizing animal facial landmarks via K nearest neighbor (kNN) search, thin plate spline warping network and fine-tuning; achieved significant improvement especially when training data are scarce.
\item Developed the holistic system in Torch and Python; obtained reasonable baseline results. 
\item Built a dataset with 3900 horse facial images and keypoint annotations; developed an annotation tool.
\end{rList}


%------------------------------------------------

{\bf Undergraduate Member -- State Key Lab of CAD \& CG, Zhejiang University} \\
\vspace{0.5em}
\textsl{Advisor: Prof. Deng Cai} \\
{\bf A Revisit on Binary Code Learning for \\Large-scale Content Based Image Retrieval (CBIR)} \hfill \textsl{May 2016 -- Present} \smallskip
\begin{rList}
\item Identified and empirically proved common insufficiencies in the experimental settings of state-of-the-art deep hashing methods.
\item Proposed a revised experimental setting for better evaluating hashing methods for CBIR tasks and made the setting public as a new benchmark dataset.
\item Conducted experiments under the revised setting to compare these deep hashing methods with traditional hashing and approximate nearest neighbor search algorithms.
\item Verified and analyzed the inferiority of these deep hashing methods.
\end{rList}

%{\em Fast kNN Graph Construction with Locality Sensitive Hashing}
{\bf EFANNA : An Extremely Fast Approximate Nearest Neighbor \\ Search Algorithm Based on kNN Graph} \hfill \textsl{Feb. 2016 -- June 2016} \smallskip
\begin{rList}
\item Contributed to the EFANNA open source C++ library and conducted comparison experiments.
\item Adopted the {\em Lanczos} algorithm, the Boost and CLAPACK library to implement the {\em Anchor Graph Hashing} and {\em Fast kNN Graph Construction with Locality Sensitive Hashing} algorithms; achieved high computational efficiency. 
\item Implemented multi-threading via OpenMP API for the EFANNA library.
\item Developed the binary code search algorithm for the EFANNA library.
\end{rList}


{\bf License Plate Recognition System} \hfill \textsl{Sept. 2015 -- Feb. 2016} \smallskip
\begin{rList}
\item Proposed a robust iterative license plate segmentation algorithm.
\item Designed and implemented a license plate segmentation system through combining my algorithm with traditional vision algorithms; achieved the error rate of 4\% on low resolution images.
\item Built a license detection system with robust skew and slant correction for better segmentation results.
%\item Built a license detection system with robust skew and slant correction to provide better source images for segmentation.
\item Wrote three literature reviews on license plate detection, segmentation and character recognition.
\end{rList}

\end{rSection}

%----------------------------------------------------------------------------------------
%	PROJECTS SECTION
%----------------------------------------------------------------------------------------
\begin{rSection}{Selected Projects}


\begin{rSubsection}{{\bf Curriculum Design Projects, Zhejiang University}}{}{Team leader}{June 2014 -- June 2015}
\item {\bf Connect Them:}  Built a novel news search engine in Python based on extensive research, which supported searching by key words \& by article, and connected semantically relevant articles; displayed the connection by charts.
\item {\bf MiniSQL:} Designed and implemented a single-user database system in C++, comprising Buffer Manager, Record Manager, Index Manager, Catalog Manager, API, and Interpreter.
\item {\bf ZCC:}  Developed a C compiler in Python, which featured compiler optimizations and error handling \& recovery; made it generate X86 assembly (runnable on real computers; no need for virtual machines).
\end{rSubsection}



\begin{rSubsection}{{\bf Student Research and Training Program (SRTP), Zhejiang University}}{}{Co-developer; Advisor: Prof. Xiaogang Jin}{March 2015 -- Nov 2016}
% Information Subscription
\item Developed Influx, an Android application, which featured a self-defined subscription function, allowing users to select and add any list-like sections on web pages to their home-made news library.
%\item Utilized open source projects, EventBus, greenDAO, and JSoup, to check the updates of the chosen sections and notify users.
\end{rSubsection}
 


\begin{rSubsection}{{\bf Computer Hardware Interest Group, Zhejiang University}}{}{Member; Instructor: Prof. Qingsong Shi}{\hspace*{\fill} March 2014 -- Sept. 2015}
\item {\bf Mine Sweeper on FPGA board}: Utilized logical circuit design to develop a salute to the classic mine sweeper game in Verilog HDL, using VGA display.
\item {\bf Single-cycle and Multi-cycle CPU on FPGA board}: Designed and implemented a single-cycle and a multi-cycle CPU with 23 basic MIPS instructions through schematic design and Verilog HDL.
\item {\bf 5-stage pipelined CPU on FPGA board}: Designed and implemented forwarding paths, branch `predict-not-taken', and interrupts in my pipelined CPU with 18 MIPS instructions.
\end{rSubsection}



\end{rSection}


%----------------------------------------------------------------------------------------
%	HONORS AND AWARDS SECTION
%----------------------------------------------------------------------------------------

\begin{rSection}{Selected Honors \& Awards}
\vspace{0.5em}
\begin{rList}
\item National Scholarship in China ({\bf 1.5\%})									\hfill 2015, 2016
\item First-Class Scholarship for Outstanding Students ({\bf 3\%})                  				\hfill 2015, 2016
\item First-Class Scholarship for Outstanding Merits ({\bf 3\%})                  					\hfill 2015, 2016
\item HE Zhijun Scholarship {\tiny (Highest �scholarship in the College of Computer Science \& Technology, Zhejiang University.)} \hfill 2016
%\item Second-Class Scholarship for Outstanding Students ({\bf 8\%})                       			\hfill 2014
%\item Second-Class Scholarship for Outstanding Merits ({\bf 8\%})                       			\hfill 2014                         
%\item Honorable Mention, Interdisciplinary Contest in Modeling Contest					\hfill 2016
%\item $2^{nd}$ Prize, Collegiate Advanced Higher Mathematics Contest of Zhejiang Province    \hfill 2014  
\item Excellent Student Awards                     						\hfill 2014
\end{rList}
\end{rSection}



%----------------------------------------------------------------------------------------
%	SKILLS SECTION
%----------------------------------------------------------------------------------------

\begin{rSection}{Skills \& Hobbies}
%\hspace*{5em}
\vspace{0.5em}
\begin{rList}
\item {\bf Hacking Skills:} Caffe, Torch, OpenCV, Python, C/C++, Matlab, Shell Script, Javascript, \LaTeX, HTML/CSS, SQL.
%\item {\bf Frameworks \& Libraries:} Caffe, Torch, OpenCV, Flann, Boost.
%\item {\bf Miscellaneous:} HTML/CSS, SQL, \LaTeX.
\item {\bf Test Scores:} TOEFL - 110, GRE - {\bf Verbal 166}, Quantitative 168, Analytical Writing 4.0.
\item {\bf Hobbies:} Mathematics, Literature, Traveling, Ping Pong, Painting, Piano.
\end{rList}
\end{rSection}


%----------------------------------------------------------------------------------------
%	Extra-Curricular SECTION
%----------------------------------------------------------------------------------------
\begin{rSection}{Extra-Curricular}

\vspace{0.5em}
\begin{rList}
\item {\bf Debate Team of School of Medicine:} Participated in the Newborn Cup Debate Competition and the Qizhen Cup Debate Competition.
\item {\bf Investigation on the National Intangible Cultural Inheritance--Northeast Errenzhuan:} Conducted field study of Errenzhuan and proposed new ways for its inheritance and promotion.
\item {\bf Member of Student Association of Science and Technology:} Managed the online GEEK station, GEEK*ZJU.
\end{rList}

\end{rSection}


\end{document}
